\subsection{几种常用的数制}

\paragraph{一、十进制}
在十进制数中,每一位有$ 0~9 $十个数码,所以计数的基数为$ 10 $。超过$ 9 $的数必须用多位数表示,其中低位和相邻高位之间的关系是“逢十进一”,故称十进制。任意一个多位的十进制数$ D $均可展开为:
\begin{equation}
    D = \sum k_{i} \times 10^{i}
    \label{十进制数}
\end{equation}
式中$ k_{i} $是第$ i $位的系数。

若以$ N $取代公式\ref{十进制数}中的$ 10 $,即可得到多位任意进制数展开式的普遍形式:
\begin{equation}
    D = \sum k_{i} N^{i}
    \label{多进制数}
\end{equation}

\paragraph{二、二进制}
目前数字电路中最广泛的是二进制,在二进制中,每一位仅有$ \mathbf{0} $和$ \mathbf{1} $两个可能的数码,所以计数基数为$ 2 $。低位和相邻高位间的进位关系是“逢二进一”,故称二进制。

根据公式\ref{多进制数},任何一个二进制数均可展开为:
\begin{equation}
    D = \sum k_{i}2^{i}
    \label{二进制数}
\end{equation}

\paragraph{三、八进制}
八进制数的每一位有$ 0~7 $八个不同的数码,计数的基数为$ 8 $。低位和相邻高位之间的进位关系是“逢八进一”。任意一个八进制数可以展开为:
\begin{equation}
    D = \sum k_{i}8^{i}
    \label{八进制数}
\end{equation}

有时采用O(Octal)下脚注,表示八进制数。

\paragraph{四、十六进制}

十六进制数的每一位有十六个不同的数码,分别用$ 0~9, A, B, C, D, E, F $表示。任意一个十六进制数均可展开为:
\begin{equation}
    D = \sum k_{i}16^{i}
    \label{十六进制数}
\end{equation}

有时采用H(Hexadecimal)下脚注表示十六进制数。