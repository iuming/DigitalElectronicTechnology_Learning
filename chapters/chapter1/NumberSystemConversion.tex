\subsection{不同数制间的转换}

\paragraph{一、二-十转换}
将二进制数转换为十进制数称为\textbf{二-十转换}。转换时只需将二进制数使用公式\ref{二进制数}进行展开即可。

\paragraph{二、十-二转换}
将十进制数转换为二进制数称为\textbf{十-二转换}。十-二转换可以分为整数部分的转换和小数部分的转换。

由公式\ref{二进制数}可知,十进制数等值的二进制数为:
\begin{equation}
    \begin{aligned}
        \left( S \right)_{10} & = \left( k_{n}2^{n} + k_{n-1}2^{n-1} + \cdots + k_{1}2^{1} + k_{0}2^{0} \right)_{2} \\
                              & = 2\left( k_{n}2^{n-1} + k_{n-1}2^{n-2} + \cdots + k_{1} \right)_{2} + k_{0}
    \end{aligned}
\end{equation}
上式表明,若将$ \left( S \right)_{10} $除以$ 2 $,则得到的商为$ k_{n}2^{n-1} + k_{n-1}2^{n-2} + \cdots + k_{1} $,而余数为$ k_{0} $。

以此类推,将每次得到的商再除以$ 2 $,就可求得二进制数的每一位了。

对于小数的转换,若$ \left( S \right)_{10} $是一个十进制的小数,对应的二进制小数为$ \left( 0.k_{-1}k_{-2}\cdots k_{-m} \right)_{2} $,根据公式\ref{二进制数}可知:
\begin{equation}
    \left( S \right)_{10} = \left( k_{-1}2^{-1} + k_{-2}2^{-2} + \cdots + k_{-m}2^{-m} \right)_{2}
\end{equation}
将上式两边同时乘以$ 2 $得到:
\begin{equation}
    2\left( S \right)_{10} = k_{-1} + \left( k_{-2}2^{-1} + k_{-3}2^{-2} + \cdots + k_{-m}2^{-m+1} \right)_{2}
    \label{小数部分十-二转换}
\end{equation}

公式\ref{小数部分十-二转换}说明,将小数$ \left( S \right)_{10} $乘以$ 2 $所得乘积的整数部分即$ k_{-1} $。

同理,将乘积的小数部分再乘以$ 2 $又可得到:
\begin{equation}
    2 \left( k_{-2}2^{-1} + k_{-3}2^{-2} + \cdots + k_{-m}2^{-m+1} \right)_{2} = k_{-2} + \left( k_{-3}2^{-1} + \cdots + k_{-m}2^{-m+2} \right)_{2}
\end{equation}
即乘积的整数部分为$ k_{-2} $。

以此类推,将每次乘$ 2 $后所得的小数部分再乘以$ 2 $,便可求出二进制小数的每一位了。

\paragraph{三、二-十六转换}
将二进制数转换为等值的十六进制数称为\textbf{二-十六转换}。

由于$ 4 $位二进制数恰好有$ 16 $个状态,而把这$ 4 $位二进制数看作一个整体时,它的进位输出正好是逢十六进一,所以只要从低位到高位将整数部分每$ 4 $位二进制数分为一组转换为等值的十六进制数,同时从高位到低位将小数部分每$ 4 $分为一组转换为等值的十六进制数,即可得到对应的十六进制数。

\paragraph{四、十六-二转换}
\textbf{十六-二转换}是将十六进制数转换为等值的二进制数。

转换时只需将十六进制数的每一位用等值的4位二进制数代替就行了。

\paragraph{五、八进制数与二进制数的转换}
将八进制数与二进制数转换的方法与十六进制数与二进制数转换的方法一样,只不过二进制数转换为八进制数只取$ 3 $位。

\paragraph{六、十六进制数与十进制数的转换}

将十六进制数转换为十进制数时可以根据公式\ref{十六进制数}进行转换,而将十进制数转换为十六进制数时,可先将十进制数转换为二进制数,再将二进制数转为十六进制数即可。