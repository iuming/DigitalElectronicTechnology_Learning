\subsection{二进制算术运算}

\subsubsection{二进制算术运算的特点}

当两个二进制数码表示两个数量大小时,它们之间可以进行数值计算,这种运算称为\textbf{算术运算}。

二进制算术运算有两个特点:
\begin{enumerate}
    \item 二进制数的乘法运算可以通过若干次的“被乘数(或零)左移一位”和“被乘数(或零)与部分积相加”这两种操作完成;
    \item 二进制数的除法运算能通过若干次的“除数右移一位”和“从被除数或余数中减去除数”这两种操作完成。
\end{enumerate}

如果能够设法将减法操作转换为某种形式的加法操作,那么加、减、乘、除运算就全部可以用“位移”和“相加”两种操作实现了。

\subsubsection{反码、补码和补码运算}
\textbf{原码:}在数字电路中,是用电路输出的高、低电平表示二进制中的$ 1 $和$ 0 $的,而二进制数的正负是在二进制数的前面增加一位符号位。符号位为$ 0 $表示这个数是正数,符号位为$ 1 $表示这个数是负数。

\textbf{补码:}对于有效数字(不包括符号位)为$ n $位的二进制数$ N $,它的补码$ \left( N \right)_{COMP} $表示方法位:
\begin{equation}
    \left( N \right)_{COMP} =
    \left\{
    \begin{aligned}
         & N \qquad  & \text{(当N为正数)} \\
         & 2^{n} - N & \text{(当N为负数)}
    \end{aligned}
    \right.
    \label{二进制补码}
\end{equation}
即正数(当符号位为$ 0 $时)的补码与原码相同,负数(当符号位为$ 1 $时)的补码等于$ 2^{n} - N $,符号位保持不变。

\textbf{反码:}二进制中的反码是这样定义的:
\begin{equation}
    \left( N \right)_{INV} =
    \left\{
    \begin{aligned}
         & N \qquad                     & \text{(当N为正数)} \\
         & \left( 2^{n} - 1 \right) - N & \text{(当N为负数)}
    \end{aligned}
    \right.
    \label{二进制反码}
\end{equation}
由上式可以知道,当$ N $为负数时,$ N + \left( N \right)_{INV} = 2^{n} - 1 $,而$ 2^{n} - 1 $是$ n $位全为$ 1 $的二进制数,因此只要将$ N $中每一位的$ 1 $改成$ 0 $、$ 0 $改成$ 1 $,就得到了$ \left( N \right)_{INV} $。

由公式\ref{二进制反码}可以得到,当$ N $为负数时,$ \left( N \right)_{INV} + 1 = 2^{n} - N $,又由公式\ref{二进制补码}可知,当$ N $为负数时,$ \left( N \right)_{COMP} = 2^{n} - N $,因此:
\begin{equation}
    \left( N \right)_{COMP} = \left( N \right)_{INV} + 1
\end{equation}
即二进制负数的补码等于它的反码加$ 1 $。

\textbf{补码运算:}若将两个加数的符号位和来自最高有效数字位的进位相加,得到的结果(舍弃产生的进位)就是和的符号。在两个同符号数相加时,它们的绝对值之和不可超过有效数字位所能表示的最大值,否则会得出错误的计算结果。