\subsection{概述}

数字电路中所处理的各种数字信号都是以\textbf{数码}形式给出的。

\begin{equation*}
    \text{不同的数码}
    \left\{
    \begin{aligned}
        \text{不同数量的大小} \\
        \text{不同的事物}     \\
        \text{事物的不同状态}
    \end{aligned}
    \right.
\end{equation*}

\textbf{数制:}多位数码中每一位的构成方法和从低位到高位的进制规则。

\begin{equation*}
    \text{数制}
    \left\{
    \begin{aligned}
        \text{二进制}   \\
        \text{十六进制} \\
        \text{十进制}   \\
        \text{八进制}
    \end{aligned}
    \right.
\end{equation*}

\textbf{算术运算:}当两个数码分别表示两个数量的大小时,可以进行数量间的加、减、乘、除运算。

\textbf{代码:}在用不同数码表示不同事物或者事物的不同状态时,这些数码只是不同事物的代号。

\textbf{码制:}在编制代码时遵循的规则。国际上通用的码制有美国信息交换标准代码(ASCII码)。