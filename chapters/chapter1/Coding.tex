\subsection{几种常用的编码}

\paragraph{一、十进制代码}
为了用二进制代码表示十进制数的$ 0~9 $这十个状态,二进制代码至少应当有$ 4 $位。$ 4 $位二进制代码一共有十六个,取其中哪十个以及如何与$ 0~9 $相对应,有许多中方案。

常用的编码规则有\textbf{8421码(BCD码)}、\textbf{余3码}、\textbf{2421码}、\textbf{5211码}、\textbf{余3循环码}。8421码中每一位的权是固定不变的,它属于恒权代码;把余3码看作$ 4 $位二进制数,它的数值要比它所表示的十进制数码多3,故称余3码;

\paragraph{二、格雷码}
格雷码(Gray Code)又称循环码,格雷码的构成方法就是每一位的状态变化都按一定的顺序循环。

与普通的二进制代码相比,格雷码的最大优点就在于当它的编码顺序依次变化时,相邻两个代码之间只有一位发生变化。这样在代码转换时就不会产生过渡“噪声”。十进制代码中的余3码就是取$ 4 $位格雷码中的十个代码组成的,它仍然具有格雷码的优点,即两个相邻代码之间仅有一位不同。

\paragraph{三、美国信息交换标准代码(ASCII)}
ASCII码是一组$ 7 $位二进制代码$ \left( b_{7}b_{6}b_{5}b_{4}b_{3}b_{2}b_{1} \right) $,共$ 128 $个,其中包括$ 0~9 $的十个代码,表示大、小写英文字母的$ 54 $个代码,$ 32 $个表示各种符号的代码以及$ 34 $个控制码。